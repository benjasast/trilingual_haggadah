% =============== Preamble ====================
\documentclass[12pt,twoside]{article}

% For XeLaTeX font loading & multilingual support
\usepackage{fontspec}
\usepackage{polyglossia}
\setdefaultlanguage{hebrew}
\setotherlanguage{english}
\setotherlanguage{spanish}

% Force explicit LTR/RTL using bidi
\usepackage{bidi}

% Hebrew font (ensure the font file exists at the specified path)
\newfontfamily\hebrewfont[
  Script=Hebrew,
  Scale=1.2,
  Path=C:/Users/SasTrakinskyB/Downloads/EzraSIL-2.51/EzraSIL2.51/
]{SILEOT.ttf}

% Latin font
\setmainfont{Libre Baskerville}

% Page geometry and styling
\usepackage[inner=3cm,outer=2.5cm,top=2.5cm,bottom=2.5cm]{geometry}
\usepackage{fancyhdr}
\pagestyle{fancy}
\fancyhf{}
\cfoot{\thepage}
\renewcommand{\headrulewidth}{0pt}

\usepackage{setspace}
\setstretch{1.2}
\usepackage{microtype}

% For side-by-side columns
\usepackage{paracol}

% For conditionals in the environment
\usepackage{etoolbox}
\newbool{firststanza}
\setbool{firststanza}{true}

% Hebrew macro: force RTL using bidi, no indent, large font
\newcommand{\HebrewText}[1]{%
  \selectlanguage{hebrew}%
  \begin{RTL}%
    \noindent\Large #1%
  \end{RTL}%
}

% English and Spanish content will be stored here:
\newcommand{\EnglishContent}{} 
\newcommand{\SpanishContent}{}

% Smart TriLingualStanza environment:
\newenvironment{TriLingualStanza}{%
  % Start: Force Hebrew on next odd page
  \clearpage
  \ifbool{firststanza}{}{%
    \cleardoublepage
  }%
  \setbool{firststanza}{false}
}{%
  % End: Move to next even page for translations
  \clearpage
  \setlength{\columnsep}{1.2cm}
  \begin{paracol}{2}
    % Left column: English; add \vspace*{0pt} to force top alignment.
    \begin{leftcolumn}
      \selectlanguage{english}%
      \normalsize%
      \noindent%
      \vspace*{0pt}%
      \EnglishContent
    \end{leftcolumn}
    \switchcolumn
    % Right column: Spanish; also force top alignment.
    \begin{rightcolumn}
      \selectlanguage{spanish}%
      \normalsize%
      \noindent%
      \vspace*{0pt}%
      \SpanishContent
    \end{rightcolumn}
  \end{paracol}
  \vspace{2em}%
}

% =============== Document Start ================
\title{הגדה של פסח}
\author{עיצוב לדוגמה}
\date{}

\begin{document}

\maketitle
\thispagestyle{empty}
\clearpage

% ---------------------------
% EXAMPLE STANZA #1
% ---------------------------
\renewcommand{\EnglishContent}{%
If He had brought us out of Egypt and had not executed judgments upon them --- \textbf{Dayenu!}\\[1em]
If He had executed judgments upon them and not upon their gods --- \textbf{Dayenu!}\\[1em]
If He had executed judgments on their gods and had not slain their firstborn --- \textbf{Dayenu!}
}
\renewcommand{\SpanishContent}{%
Si Él nos hubiera sacado de Egipto y no hubiera ejecutado juicios contra ellos, \textbf{Dayenu!}\\[1em]
Si Él hubiera ejecutado juicios contra ellos y no contra sus dioses, \textbf{¡Dayenu!}\\[1em]
Si Él hubiera ejecutado juicios contra sus dioses y no hubiera matado a sus primogénitos, \textbf{¡Dayenu!}
}
\begin{TriLingualStanza}
\HebrewText{%
אִלּוּ הוֹצִיאָנוּ מִמִּצְרַיִם וְלֹא עָשָׂה בָהֶם שְׁפָטִים --- \textbf{דַּיֵּנוּ}\\[0.8em]
אִלּוּ עָשָׂה בָהֶם שְׁפָטִים וְלֹא עָשָׂה בֵאלֹהֵיהֶם --- \textbf{דַּיֵּנוּ}\\[0.8em]
אִלּוּ עָשָׂה בֵאלֹהֵיהֶם וְלֹא הָרַג אֶת בְּכוֹרֵיהֶם --- \textbf{דַּיֵּנוּ}
}
\end{TriLingualStanza}

% ---------------------------
% EXAMPLE STANZA #2
% ---------------------------
\renewcommand{\EnglishContent}{%
If He had given us the Torah and not brought us into the Land of Israel --- \textbf{Dayenu!}\\[1em]
If He had fed us the manna in the desert and not given us the Sabbath --- \textbf{Dayenu!}
}
\renewcommand{\SpanishContent}{%
Si Él nos hubiera dado la Torá y no nos hubiera llevado a la Tierra de Israel, \textbf{¡Dayenu!}\\[1em]
Si Él nos hubiera alimentado con el maná en el desierto y no nos hubiera dado el Shabat, \textbf{¡Dayenu!}
}
\begin{TriLingualStanza}
\HebrewText{%
אִלּוּ נָתַן לָנוּ אֶת הַתּוֹרָה וְלֹא הִכְנִיסָנוּ לְאֶרֶץ יִשְׂרָאֵל --- \textbf{דַּיֵּנוּ}\\[0.8em]
אִלּוּ הִנְהִיג לָנוּ מָן בַּמִּדְבָּר וְלֹא נָתַן לָנוּ אֶת הַשַּׁבָּת --- \textbf{דַּיֵּנוּ}
}
\end{TriLingualStanza}

\end{document}
